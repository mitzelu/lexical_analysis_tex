\section*{Abstract}
\pagenumbering{gobble}
Thesis \textbf{Lexical Analysis of Social Media} presented by Leahu Luminita was written in English. It has X figures, X listings, X tables, and X references. The report consists of introduction, 4 chapters, and conclusions.

The thesis aims to research different tools and approaches of lexical analysis in context of extracting useful information from social media. In this context, Finduber project was developed. It represents a platform that uses a lexial based search engine to fetch influential bloggers from Youtube channel that. 

The first chapter represents an overview of programs, methods and theory used for development. Also are analysed existed products.

In the second chapter the system is analyzed from the architectural point of view. The artifacts of Finduber system using UML diagrams are specified and documented in this chapter.

Third chapter contains will be analysed and described every step of the implementation process, together with the used technologies and code snippets. Each technologie and 

In chapter four Finduber project is analyzed in economic terms. The analysis is done to assess the opportunity of the project by considering the benefits compared to the costs. Besides financial indicators, other additional aspects as market influences, social and envrironmental costs are considered. 

Finduber represents an attempt to offer a marketing tool for brands that want another way of promoting the products based on machine learning algorithms. It does not have many features but it has the right infrastructure for future development. It also proved that a data journalism software has a lot of potential on inexistent local market.

Keywords: lexical analysis; text mining; social media; machine learning.