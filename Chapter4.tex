\section{Economic Analysis}
\phantomsection

\subsection{Project description}

Nowdays social media become an important marketing tool. Beside a big network where everyone can share information and express their thougts, social media channels represent a good place for advertising different types of products and services. These special types of posts are called blogs. In this context people start to use social media as a source of inspiration regarding questions like: how to dress? what to cook? how to loose weight? what products to use for skin? Together with people interest about blogs, increased and interest of companies who provides products and services.

MyApp represents a tool which will help companies to find influential bloggers according to their preferences.

\subsection{Project time schedule}

For a better development of the project a well defined time schedule is needed. The most suitable approach for MyApp application is agile project management, because it focuses on continuous improvement. It consists of several iterations, each interation has 5 steps: planning, research, development, testing and deployment.

\subsubsection{Objective determination}

During the first step of agile project management -- planning -- a set of objectives are established in order to determine what the project is supposed to accomplish when ended. Without a well defined objectives it is impossible to evaluate the results and to plan the activities for achieving them. This is the reason why objectives should be SMART. 

The main goal of the project is to give stakeholders a list of influential bloggers, according to their requirements, with estimated impact regarding product visibility.

The key objectives that should be meet in order to achieve the goal are:

\begin{itemize}

\item[--] \textit{To organize information from the database into well defined categories.} This objective is crucial, because social media content represents an unstructured data and it is hard to analyse it in the raw form. Organizing information by categories will lead to a better processing.

\item[--] \textit{To create dependeces between categories.} Not all the words can be set as an independent category. That's why a category tree, which includes subcategories, should be created for each category.

\item[--] \textit{To develop a search mechanism according to specified requirements.} The results obtained after search should be as specific as possible, meeting all requirements as stated by user.

\item[--] \textit{To filter the resulting list and to keep only the influential bloggers.} Since the big amount of data, the resulting list after search should be huge, containing also bloggers not so influnetial. A filter according to numbers of views, reviews and numbers of subscribes have to be done.

\item[--] \textit{To estimate the impact of choosing that specific bloger as a promoter.} Beside a list of names, the application should offer an impact estimation regarding product visibility.

\end{itemize}

\subsubsection{Time schedule establishment}

Project scheduling consists of three main parts: what activity needs to be performed, the number of days in which it should be performed and which people are responsible for it. The time schedule is divided according to specified agile project management steps. Thus, for planning and researching phases there is not a strict ammount of time and a well determined splitting, because the requirements may change and the process of analysing should be repeated. The same thing can not be assumed for development step, where the process is splited up in smaller tasks with a defined period of days. To be able to compute the total duration of the project the formula \eqref{eq:duration} is used.

\begin{equation} \label{eq:duration}
 D_T = D_F - D_S + T_R,
\end{equation}

\noindent
where $D_T$ is the duration, $D_F$ -- the finish date, $D_S$ -- the start date and $T_R$ -- reserve time. The first iteration of the project schedule is presented in Table \ref{table:schedule}. To define the people working on the project the following notations are used: PM -- project manager, SA -- system architect, SM -- sales manager, D -- developer.

\begin{table}[!ht]
\begin{center}
\caption{Time schedule}
\renewcommand{\arraystretch}{2}
\begin{tabular}{| c | >{\centering\arraybackslash}p{7.5cm}  | >{\centering\arraybackslash}p{5cm} | c |}
\hline
\textbf{Nr} & \textbf{Activity Name} & \textbf{Duration (days)} & \textbf{People involved}  \\
\hline
1 & Project specification and business processes & 6 & PM, SA, SM, D  \\
\hline
2 & Analysis of market & 10 & PM, SA  \\
\hline
3 & Analysis of the domain & 14 & SA, D  \\
\hline
4 & Requirements analysis and catalogue & 5 & PM, SA, D  \\
\hline
5 & System design (UML) & 12 & PM, SA, D  \\
\hline
6 & Database design & 7 & PM, SA, D \\
\hline
7 & Specification and analysis of architectue and technologies & 20 & PM, SA, D \\
\hline
8 & End-user application development & 30 & PM, SA, D, SM  \\
\hline
9 & Validation of results & 10 & PM, SA, D, SM  \\
\hline
10 & Documentation & 7 & D  \\
\hline
11 & Deployment and testing & 10 & PM, SA, D  \\
\hline
12 & Active marketing & 7 & SM  \\
\hline
13 & Total time to finish the system & 138 &  \\
\hline
\end{tabular}
\label{table:schedule}
\vspace{-2.5em}
\end{center}
\end{table}

Table \ref{table:schedule} describes in the main activities during first iteration of the project together with number of days allocated per activity. By summing these days we get a total amount of time of 138 days nedded to complete the project.

\subsection{Economic motivation}

In this section the project is evaluated from an economic point of view, which implies a measure of its net benefits in monetary terms (MDL (Moldavian lei) currency) by using real or estimated market prices. During analysis all expenditures incurred under the project like: tangible and intangible assets, direct and salary expenses,  and revenues resulting from it are taken into acount.

\subsubsection{Tangible and intangible asset expenses}

In Table \ref{table:tangible_assets} are listed tangible assets and their expenses. The term tangible asset means an asset that has a physical form, such as machinery.

\begin{table}[!hb]
\begin{center}
\caption{Tangible asset expenses}
\renewcommand{\arraystretch}{2}
\begin{tabular}{| >{\centering\arraybackslash}p{1.7cm}  | >{\centering\arraybackslash}p{5cm} | >{\centering\arraybackslash}p{2.7cm} | >{\centering\arraybackslash}p{2cm} | c | >{\centering\arraybackslash}p{5em}|}
\hline
\textbf{Material} & \textbf{Specification} & \textbf{Measurement unit} & \textbf{Price per unit (MDL)} & \textbf{Quantity} & \textbf{Sum (MDL)}\\
\hline
Lenovo ideapad & i5 & Unit & 12000 & 1 &  \multicolumn{1}{r|}{12000}\\
\hline
\multicolumn{5}{|r|}{Total} & \multicolumn{1}{r|}{12000}\\
\hline
\end{tabular}
\label{table:tangible_assets}
\end{center}
\vspace{-1.3em}
\end{table}

The budget for the required intangible assets, i.e., nonphysical assets, such as patents, trademarks, copyrights brand recognition, is shown in Table \ref{table:intangible_assets}.

\begin{table}[!hb]
\begin{center}
\caption{Intangible asset expenses}
\renewcommand{\arraystretch}{2}
\begin{tabular}{| c | >{\centering\arraybackslash}p{5cm} | >{\centering\arraybackslash}p{2.7cm} | >{\centering\arraybackslash}p{2cm} | c | >{\centering\arraybackslash}p{5em}|}
\hline
\textbf{Material} & \textbf{Specification} & \textbf{Measurement unit} & \textbf{Price per unit (MDL)} & \textbf{Quantity} & \textbf{Sum (MDL)} \\
\hline
License & Enterprise Architect Desktop Edition License & Unit & 1900 & 3 & \multicolumn{1}{r|}{5700} \\
\hline
\multicolumn{5}{|r|}{Total} & \multicolumn{1}{r|}{5700}\\
\hline
\end{tabular}
\label{table:intangible_assets}
\vspace{-1em}
\end{center}
\end{table}

Direct expenses are presented in Table \ref{table:direct_expenses}. It includes purchase of raw materials used during planing, analysing and meetings. These expenses are indispensable in every project at any stage.

So the total amount of direct expenses in MDL is:

\begin{equation}
 T_e = 12000 + 5700 + 1017 = 18717
\end{equation}

\begin{table}[!hb]
\begin{center}
\caption{Direct expenses}
\renewcommand{\arraystretch}{2}
\begin{tabular}{| >{\centering\arraybackslash}p{7.5em} | >{\centering\arraybackslash}p{8em} | >{\centering\arraybackslash}p{7em} | >{\centering\arraybackslash}p{5em} | >{\centering\arraybackslash}p{5em} | r |}
\hline
\textbf{Material} & \textbf{Specification} & \textbf{Measurement unit} & \textbf{Price per unit (MDL)} & \textbf{Quantity} & \multicolumn{1}{>{\centering\arraybackslash}p{5em}|}{\textbf{Sum (MDL)}}\\
\hline
Whiteboard & Universal Dry Erase Board & Unit & 500 & 1 & 500 \\
\hline
Paper & A4 & 250 sheets & 60 & 1 & 60 \\
\hline
Marker & Whiteboard marker & Unit & 15 & 5 & 75 \\
\hline
Eraser & Whiteboard eraser & Unit & 60 & 2 & 120 \\
\hline
Pen & Blue pen & Unit & 5 & 10 & 50 \\
\hline
Flipchart paper & A1 & 80sheets & 75 & 2 & 150 \\
\hline
Tape & Paper tape & Unit & 6 & 2 & 12 \\
\hline
Marker & Permanent marker & Unit & 10 & 5 & 50 \\
\hline
\multicolumn{5}{|r|}{Total} & 1017 \\
\hline
\end{tabular}
\label{table:direct_expenses}
\vspace{-1.5em}
\end{center}
\end{table}

\subsubsection{Salary expenses}

This section includes the budget which is allocated for employee remuneration. As specified above on the project will work 4 people and according to their position a different amount of money will be paid per day, as shown in Table {table:salaries}.

\begin{table}[!ht]
\begin{center}
\caption{Salary expenses}
\renewcommand{\arraystretch}{2}
\begin{tabular}{| >{\centering\arraybackslash}p{8em} | >{\centering\arraybackslash}p{8em} | >{\centering\arraybackslash}p{8em} | r |}
\hline
\textbf{Employee} & \textbf{Work fund (days)} & \textbf{Salary per day (MDL)} & \multicolumn{1}{>{\centering\arraybackslash}p{5em}|}{\textbf{Salary fund (MDL)}}\\
\hline
Project Manager & 120 & 350 & 42000 \\
\hline 
System Architect & 90 & 400 & 36000\\
\hline
Sales Manager & 60 & 250 & 15000\\
\hline
Developer & 120 & 330 & 39600\\
\hline
\multicolumn{3}{|r|}{Total} & 132600\\
\hline
\end{tabular}
\label{table:salaries}
\vspace{-2.5em}
\end{center}
\end{table}

Since, the project involves employees it is necessary to compute how much should be paid to medical insurance, social services fund and the total work expenses.

The medical insurance fund is calculated as:

\begin{equation}
\begin{split}
 MI &= F_{re} \cdot T_{mi}\\ 
    &= 132600 \cdot 0.035\\ 
    &= 4641,
 \end{split}
\end{equation}

\noindent
where $T_{mi}$ is the mandatory medical insurance tax and this year it is $3.5\%$. 

The social fund for this year is 23\%, thus the salary expenses are computed as follows:

\begin{equation}
\begin{split}
 FS &= F_{re} \cdot T_{fs} \\
    &= 132600 \cdot 0.23 \\
    &= 30498,
\end{split}
\end{equation}
\noindent
where $FS$ is the salary expense, $F_{re}$ is the salary expense fund and $T_{fs}$ is the social service tax.

Having all the calculations done for social service tax and medical insurance tax, it is possible to found out total work expense using the relation \eqref{eq:total_work_expense}

\begin{equation}\label{eq:total_work_expense}
\begin{split}
 WEF &= F_{re} + FS + MI\\
     &= 132600 + 30498 + 4641\\
     &= 167739,
\end{split}
\end{equation}

\noindent
where $WEF$ is the work expense fund, FS -- the social fund and MI -- the medical insurance fund.

\subsection{Individual person salary}



